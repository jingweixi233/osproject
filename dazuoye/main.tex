% !BIB TS-program = biber
\documentclass[a4paper,oneside,12pt]{article}
\usepackage{BUPTthesisbachelor}
\usepackage{setspace}
\usepackage{graphicx}

\usepackage[final]{pdfpages}

\usepackage{listings}
\usepackage{xcolor}

\usepackage{CJK}         % CJK 中文支持
\usepackage{fancyhdr}
\usepackage{amsmath,amsfonts,amssymb,graphicx}    % EPS 图片支持
\usepackage{subfigure}   % 使用子图形
\usepackage{indentfirst} % 中文段落首行缩进
\usepackage{bm}          % 公式中的粗体字符(用命令\boldsymbol)
\usepackage{multicol}    % 正文双栏

\usepackage{abstract} 

\usepackage{amssymb}
\usepackage{bm}

\usepackage{algorithm}  
\usepackage{algorithmicx}  
\usepackage{algpseudocode}  
\lstdefinestyle{sharpc}{language=[Sharp]C, frame=lrtb, rulecolor=\color{blue!80!black}}

\title{\huge{The Description of Project}}
\author{Name: Xi Jingwei  \quad Student ID:517030910116 \quad   Email:jingweixi@sjtu.edu.cn}
\date{}  % 这一行用来去掉默认的日期显示
%%%%%%%%%%%%%%%%%%%%%%%%% Begin Documents %%%%%%%%%%%%%%%%%%%%%%%%%%
%中文摘要
\begin{document}
\maketitle
\setlength{\oddsidemargin}{0.5cm}
\setlength{\evensidemargin}{\oddsidemargin}
\setlength{\textwidth}{15cm}
\vspace{-.8cm}


%% !BIB TS-program = biber
\documentclass[a4paper,oneside,12pt]{article}
\usepackage{BUPTthesisbachelor}
\usepackage{setspace}
\usepackage{graphicx}

\usepackage[final]{pdfpages}

\usepackage{listings}
\usepackage{xcolor}
\usepackage{setspace}
\usepackage{CJK}         % CJK 中文支持
\usepackage{fancyhdr}
\usepackage{amsmath,amsfonts,amssymb,graphicx}    % EPS 图片支持
\usepackage{subfigure}   % 使用子图形
\usepackage{indentfirst} % 中文段落首行缩进
\usepackage{bm}          % 公式中的粗体字符(用命令\boldsymbol)
\usepackage{multicol}    % 正文双栏
\usepackage{verbatim}
\usepackage{abstract} 

\usepackage{amssymb}
\usepackage{bm}
\usepackage{fontspec}

\usepackage{algorithm}  
\usepackage{algorithmicx}  
\usepackage{algpseudocode}

\usepackage{graphicx}


\lstdefinestyle{sharpc}{language=[Sharp]C, frame=lrtb, rulecolor=\color{blue!80!black}}

\title{\huge{The Description of Project}}
\author{Name: Xi Jingwei  \quad Student ID:517030910116 \quad   Email:jingweixi@sjtu.edu.cn}
\date{}  % 这一行用来去掉默认的日期显示
%%%%%%%%%%%%%%%%%%%%%%%%% Begin Documents %%%%%%%%%%%%%%%%%%%%%%%%%%
%中文摘要

\begin{document}
\maketitle
\setlength{\oddsidemargin}{0.5cm}
\setlength{\evensidemargin}{\oddsidemargin}
\setlength{\textwidth}{15cm}
\vspace{-.8cm}


%% !BIB TS-program = biber
\documentclass[a4paper,oneside,12pt]{article}
\usepackage{BUPTthesisbachelor}
\usepackage{setspace}
\usepackage{graphicx}

\usepackage[final]{pdfpages}

\usepackage{listings}
\usepackage{xcolor}
\usepackage{setspace}
\usepackage{CJK}         % CJK 中文支持
\usepackage{fancyhdr}
\usepackage{amsmath,amsfonts,amssymb,graphicx}    % EPS 图片支持
\usepackage{subfigure}   % 使用子图形
\usepackage{indentfirst} % 中文段落首行缩进
\usepackage{bm}          % 公式中的粗体字符(用命令\boldsymbol)
\usepackage{multicol}    % 正文双栏
\usepackage{verbatim}
\usepackage{abstract} 

\usepackage{amssymb}
\usepackage{bm}
\usepackage{fontspec}

\usepackage{algorithm}  
\usepackage{algorithmicx}  
\usepackage{algpseudocode}

\usepackage{graphicx}


\lstdefinestyle{sharpc}{language=[Sharp]C, frame=lrtb, rulecolor=\color{blue!80!black}}

\title{\huge{The Description of Project}}
\author{Name: Xi Jingwei  \quad Student ID:517030910116 \quad   Email:jingweixi@sjtu.edu.cn}
\date{}  % 这一行用来去掉默认的日期显示
%%%%%%%%%%%%%%%%%%%%%%%%% Begin Documents %%%%%%%%%%%%%%%%%%%%%%%%%%
%中文摘要

\begin{document}
\maketitle
\setlength{\oddsidemargin}{0.5cm}
\setlength{\evensidemargin}{\oddsidemargin}
\setlength{\textwidth}{15cm}
\vspace{-.8cm}


%% !BIB TS-program = biber
\documentclass[a4paper,oneside,12pt]{article}
\usepackage{BUPTthesisbachelor}
\usepackage{setspace}
\usepackage{graphicx}

\usepackage[final]{pdfpages}

\usepackage{listings}
\usepackage{xcolor}
\usepackage{setspace}
\usepackage{CJK}         % CJK 中文支持
\usepackage{fancyhdr}
\usepackage{amsmath,amsfonts,amssymb,graphicx}    % EPS 图片支持
\usepackage{subfigure}   % 使用子图形
\usepackage{indentfirst} % 中文段落首行缩进
\usepackage{bm}          % 公式中的粗体字符(用命令\boldsymbol)
\usepackage{multicol}    % 正文双栏
\usepackage{verbatim}
\usepackage{abstract} 

\usepackage{amssymb}
\usepackage{bm}
\usepackage{fontspec}

\usepackage{algorithm}  
\usepackage{algorithmicx}  
\usepackage{algpseudocode}

\usepackage{graphicx}


\lstdefinestyle{sharpc}{language=[Sharp]C, frame=lrtb, rulecolor=\color{blue!80!black}}

\title{\huge{The Description of Project}}
\author{Name: Xi Jingwei  \quad Student ID:517030910116 \quad   Email:jingweixi@sjtu.edu.cn}
\date{}  % 这一行用来去掉默认的日期显示
%%%%%%%%%%%%%%%%%%%%%%%%% Begin Documents %%%%%%%%%%%%%%%%%%%%%%%%%%
%中文摘要

\begin{document}
\maketitle
\setlength{\oddsidemargin}{0.5cm}
\setlength{\evensidemargin}{\oddsidemargin}
\setlength{\textwidth}{15cm}
\vspace{-.8cm}


%\input{main.cfg}    % Main ?????????????????????????????????????????items 
%\include{abstract}  % Abstract
%%%%%%%%%%%%%%%%%%%%%%%%%%%%% Main Area %%%%%%%%%%%%%%%%%%%%%%%%%%%%
\setlength{\oddsidemargin}{-.5cm}  % 3.17cm - 1 inch
\setlength{\evensidemargin}{\oddsidemargin}
\setlength{\textwidth}{17.00cm}

\section{Problem1}
\subsection{Files}
\subsubsection{ptree.c}
Function static int ptree(struct prinfo *buf, int *nr)

Call function $dfs$ and use read\_lock and read\_unlock to protect the data.Then copy data from task[] to buf.

Function void $dfs$(struct task\_struct start, int deep)

Search all processes and store information in $dfs$ order.

\subsubsection{Makefile}

Use Makefile to compile the ptree.c file.

\subsection{Process}

a. I searched a lot of information of task\_struct and list function in Linux kernel on the internet. Then I know how to use the function list\_entry and list\_for\_each.

b. When I coded the $dfs$ function, I met many problems. I don't know how to depth-first-search a tree without knowing all the children of parent. Then I search on Google and I find node -> children will point to the head of children list. And I use list\_for\_each function to depth-first-search the process tree.

c. At problem1, I don't know the difference between struct task\_struct and list\_head. And I didn't know how to transfer them. With the help of friends, I learn to use list\_entry((\&node , struct task\_struct, sibling) to transfer them.




\section{Problem2}
\subsection{Files}
\subsubsection{ptree.c}

The program will call ptree function in kernel and print the entire process tree (in DFS order) using tabs to indent children with respect to their parents.

\subsubsection{Android.mk}

The make file for project.

\subsection{Process}

At first, I used syscall 391 in the program, but I could not make it work in kernel and I can not find the problem. With the help of TAs, I changed syscall from 391 to 356 and then it worked. 

\subsection{Result}

\begin{lstlisting}
The number of task is 59!
Print start
swapper, 0, 0, 0, 1, 0, 0
	init, 1, 1, 0, 45, 2, 0
		ueventd, 45, 1, 1, 0, 61, 0
		logd, 61, 1, 1, 0, 62, 1036
		vold, 62, 1, 1, 0, 68, 0
		healthd, 68, 1, 1, 0, 69, 0
		lmkd, 69, 1, 1, 0, 70, 0
		servicemanager, 70, 1, 1, 0, 71, 1000
		surfaceflinger, 71, 1, 1, 0, 73, 1000
		qemud, 73, 1, 1, 0, 76, 0
		sh, 76, 1, 1, 0, 77, 2000
		adbd, 77, 1, 1, 197, 78, 0
			sh, 197, 1, 77, 370, 1, 0
				ptreeARM, 370, 0, 197, 0, 1, 0
		netd, 78, 1, 1, 371, 79, 0
			sh, 371, 0, 78, 0, 1, 0
		debuggerd, 79, 1, 1, 0, 80, 0
		rild, 80, 1, 1, 0, 81, 1001
		drmserver, 81, 1, 1, 0, 82, 1019
		mediaserver, 82, 1, 1, 0, 83, 1013
		installd, 83, 1, 1, 0, 84, 0
		keystore, 84, 1, 1, 0, 85, 1017
		main, 85, 1, 1, 235, 86, 0
			system_server, 235, 1, 85, 0, 1, 1000
		gatekeeperd, 86, 1, 1, 0, 89, 1000
		perfprofd, 89, 1, 1, 0, 90, 0
		fingerprintd, 90, 1, 1, 0, 122, 1000
		bootanimation, 122, 1, 1, 0, 1, 1003
	kthreadd, 2, 1, 0, 3, 0, 0
		ksoftirqd/0, 3, 1, 2, 0, 4, 0
		kworker/0:0, 4, 1, 2, 0, 5, 0
		kworker/u:0, 5, 1, 2, 0, 6, 0
		khelper, 6, 1, 2, 0, 7, 0
		sync_supers, 7, 1, 2, 0, 8, 0
		bdi-default, 8, 1, 2, 0, 9, 0
		kblockd, 9, 1, 2, 0, 10, 0
		rpciod, 10, 1, 2, 0, 11, 0
		kworker/0:1, 11, 1, 2, 0, 12, 0
		kswapd0, 12, 1, 2, 0, 13, 0
		fsnotify_mark, 13, 1, 2, 0, 14, 0
		crypto, 14, 1, 2, 0, 25, 0
		kworker/u:1, 25, 1, 2, 0, 30, 0
		mtdblock0, 30, 1, 2, 0, 35, 0
		mtdblock1, 35, 1, 2, 0, 40, 0
		mtdblock2, 40, 1, 2, 0, 41, 0
		binder, 41, 1, 2, 0, 42, 0
		deferwq, 42, 1, 2, 0, 43, 0
		kworker/u:2, 43, 1, 2, 0, 44, 0
		mmcqd/0, 44, 1, 2, 0, 47, 0
		jbd2/mtdblock0-, 47, 1, 2, 0, 48, 0
		ext4-dio-unwrit, 48, 1, 2, 0, 51, 0
		flush-31:1, 51, 1, 2, 0, 53, 0
		jbd2/mtdblock1-, 53, 1, 2, 0, 54, 0
		ext4-dio-unwrit, 54, 1, 2, 0, 57, 0
		flush-31:2, 57, 1, 2, 0, 59, 0
		jbd2/mtdblock2-, 59, 1, 2, 0, 60, 0
		ext4-dio-unwrit, 60, 1, 2, 0, 63, 0
		kworker/0:2, 63, 1, 2, 0, 94, 0
		kauditd, 94, 1, 2, 0, 0, 0
Print end


\end{lstlisting}

\section{Problem3}
\subsection{Files}
\subsubsection{process.c}

The program enerate a new process and output “StudentIDParent” with PID, then generates its children process output “StudentIDChild” with PID.

And in child process it will execute ptree.

\subsubsection{Android.mk}

The make file for project.

\subsection{Process}

I searched on the internet for how to get the process id, then I get know that getpid() function can get the id of process. And after searching the relationship of parent process and child process, I know the pid = fork() in parent process will return the child process id.

\subsection{Result}

\section{Problem3}
\subsection{Files}
\subsubsection{process.c}

The program enerate a new process and output “StudentIDParent” with PID, then generates its children process output “StudentIDChild” with PID.

And in child process it will execute ptree.

\subsubsection{Android.mk}

The make file for project.

\subsection{Process}

I searched on the internet for how to get the process id, then I get know that getpid() function can get the id of process. And after searching the relationship of parent process and child process, I know the pid = fork() in parent process will return the child process id.

\subsection{Result}

\begin{lstlisting}
517030910116 Parent pid = 701
517030910116 Child pid = 702
The number of task is 59!
Print start
swapper, 0, 0, 0, 1, 0, 0
	init, 1, 1, 0, 45, 2, 0
		ueventd, 45, 1, 1, 0, 61, 0
		logd, 61, 1, 1, 0, 62, 1036
		vold, 62, 1, 1, 0, 68, 0
		healthd, 68, 1, 1, 0, 69, 0
		lmkd, 69, 1, 1, 0, 70, 0
		servicemanager, 70, 1, 1, 0, 71, 1000
		surfaceflinger, 71, 1, 1, 0, 73, 1000
		qemud, 73, 1, 1, 0, 76, 0
		sh, 76, 1, 1, 0, 77, 2000
		adbd, 77, 1, 1, 197, 78, 0
			sh, 197, 1, 77, 370, 1, 0
				ptreeARM, 370, 0, 197, 0, 1, 0
		netd, 78, 1, 1, 371, 79, 0
			sh, 371, 0, 78, 0, 1, 0
		debuggerd, 79, 1, 1, 0, 80, 0
		rild, 80, 1, 1, 0, 81, 1001
		drmserver, 81, 1, 1, 0, 82, 1019
		mediaserver, 82, 1, 1, 0, 83, 1013
		installd, 83, 1, 1, 0, 84, 0
		keystore, 84, 1, 1, 0, 85, 1017
		main, 85, 1, 1, 235, 86, 0
			system_server, 235, 1, 85, 0, 1, 1000
		gatekeeperd, 86, 1, 1, 0, 89, 1000
		perfprofd, 89, 1, 1, 0, 90, 0
		fingerprintd, 90, 1, 1, 0, 122, 1000
		bootanimation, 122, 1, 1, 0, 1, 1003
	kthreadd, 2, 1, 0, 3, 0, 0
		ksoftirqd/0, 3, 1, 2, 0, 4, 0
		kworker/0:0, 4, 1, 2, 0, 5, 0
		kworker/u:0, 5, 1, 2, 0, 6, 0
		khelper, 6, 1, 2, 0, 7, 0
		sync_supers, 7, 1, 2, 0, 8, 0
		bdi-default, 8, 1, 2, 0, 9, 0
		kblockd, 9, 1, 2, 0, 10, 0
		rpciod, 10, 1, 2, 0, 11, 0
		kworker/0:1, 11, 1, 2, 0, 12, 0
		kswapd0, 12, 1, 2, 0, 13, 0
		fsnotify_mark, 13, 1, 2, 0, 14, 0
		crypto, 14, 1, 2, 0, 25, 0
		kworker/u:1, 25, 1, 2, 0, 30, 0
		mtdblock0, 30, 1, 2, 0, 35, 0
		mtdblock1, 35, 1, 2, 0, 40, 0
		mtdblock2, 40, 1, 2, 0, 41, 0
		binder, 41, 1, 2, 0, 42, 0
		deferwq, 42, 1, 2, 0, 43, 0
		kworker/u:2, 43, 1, 2, 0, 44, 0
		mmcqd/0, 44, 1, 2, 0, 47, 0
		jbd2/mtdblock0-, 47, 1, 2, 0, 48, 0
		ext4-dio-unwrit, 48, 1, 2, 0, 51, 0
		flush-31:1, 51, 1, 2, 0, 53, 0
		jbd2/mtdblock1-, 53, 1, 2, 0, 54, 0
		ext4-dio-unwrit, 54, 1, 2, 0, 57, 0
		flush-31:2, 57, 1, 2, 0, 59, 0
		jbd2/mtdblock2-, 59, 1, 2, 0, 60, 0
		ext4-dio-unwrit, 60, 1, 2, 0, 63, 0
		kworker/0:2, 63, 1, 2, 0, 94, 0
		kauditd, 94, 1, 2, 0, 0, 0
Print end

\end{lstlisting}
\section{Problem4}
\subsection{Files}
\subsubsection{server.c}

Function int main()

The server will listen to the port. If there is a client want to connect, it will connect to the client and create a new thread to call serve function.

Function void *serve(void *clientfd)

It will judge whether server could serve client or not. If the count of client that server served concurrently is less than two, the server can receive the client message and change it. Then it will send the message changed to the client.

\subsubsection{client.c}

The program client.c will send message to server and receive the message changed by server.

\subsection{Process}

1. I met a lot of problems in server problem. At first, I don't know the how to create thread and how to deal with critical section problem. I find answers on the internet and use pthread and mutex.

2. I use two mutex. One mutex is for variable count which is the number of client which are served by server. The other mutex is to make new coming clients to wait if there are two clients served concurrently.

\subsection{Result}

When client1 and client2 are served by server, client3 need to wait.

\includegraphics[scale=0.3]{1.png}

When client1 quit and end the service, the client3 will be served by server.

\includegraphics[scale=0.3]{2.png}

\end{document}    % Main ?????????????????????????????????????????items 
%%%%%%%%%%%%%%%%%%%%%%%%%%%%%%%%%%%%%%%%%%%%%%%%%%%%%%%%%%%%%%%%%%%%%
%                                                                  %
%   Modified by Bing Hsu <hello@antinucleon.com> 2013              %
%   Forked From (c) 2010 - 2011 Caspar Zhang <casparant@gmail.com> %
%                                                                  %
%   This copyrighted material is made available to anyone wishing  %
%   to use, modify, copy, or redistribute it subject to the terms  %
%   and conditions of the GNU General Public License version 2.    %
%                                                                  %
%   This program is distributed in the hope that it will be        %
%   useful, but WITHOUT ANY WARRANTY; without even the implied     %
%   warranty of MERCHANTABILITY or FITNESS FOR A PARTICULAR        %
%   PURPOSE. See the GNU General Public License for more details.  %
%                                                                  %
%   You should have received a copy of the GNU General Public      %
%   License along with this program; if not, write to the Free     %
%   Software Foundation, Inc., 51 Franklin Street, Fifth Floor,    %
%   Boston, MA 02110-1301, USA.                                    %
%                                                                  %
%%%%%%%%%%%%%%%%%%%%%%%%%%%%%%%%%%%%%%%%%%%%%%%%%%%%%%%%%%%%%%%%%%%%

\input{abstract.cfg}

% 中文摘要
\begin{titlepage}
    \begin{spacing}{1.05}
        \centering
        \parbox[c]{.6\textwidth}{\thesistitlefont{\thesistitle}}
    \end{spacing}

    \begin{spacing}{1.6}
        \centering
        \sanhao\quad{} \\ 
        \abscnname{517030910116\quad{}\quad{}\quad{}席经纬\quad{}\quad{}\quad{}\quad{}F1703301} \\ 
        \xiaosanhao\quad{}
    \end{spacing}


    \begin{spacing}{0.8}
        \centering
        \sanhao\quad{} \\ 
        \abscnname{摘\quad{}要} \\ 
        \xiaosanhao\quad{}
    \end{spacing}
    
    \normalsize

    \abstractcn
    
    \quad{}

    \abscnkey{关键词}\quad{}%
    \abscnkeys{\abscnkeyone\quad{}%
                      \abscnkeytwo\quad{}%
                      \abscnkeythree\quad{}%
                      }%

    \begin{spacing}{1.6}
        \centering
        \sanhao\quad{} \\ 
        \abscnname{引\quad{}言} \\ 
        \xiaosanhao\quad{}
    \end{spacing}
    
    \normalsize

    \abstractcns
    
    \quad{}

\end{titlepage}
  % Abstract
%%%%%%%%%%%%%%%%%%%%%%%%%%%%% Main Area %%%%%%%%%%%%%%%%%%%%%%%%%%%%
\setlength{\oddsidemargin}{-.5cm}  % 3.17cm - 1 inch
\setlength{\evensidemargin}{\oddsidemargin}
\setlength{\textwidth}{17.00cm}

\section{Problem1}
\subsection{Files}
\subsubsection{ptree.c}
Function static int ptree(struct prinfo *buf, int *nr)

Call function $dfs$ and use read\_lock and read\_unlock to protect the data.Then copy data from task[] to buf.

Function void $dfs$(struct task\_struct start, int deep)

Search all processes and store information in $dfs$ order.

\subsubsection{Makefile}

Use Makefile to compile the ptree.c file.

\subsection{Process}

a. I searched a lot of information of task\_struct and list function in Linux kernel on the internet. Then I know how to use the function list\_entry and list\_for\_each.

b. When I coded the $dfs$ function, I met many problems. I don't know how to depth-first-search a tree without knowing all the children of parent. Then I search on Google and I find node -> children will point to the head of children list. And I use list\_for\_each function to depth-first-search the process tree.

c. At problem1, I don't know the difference between struct task\_struct and list\_head. And I didn't know how to transfer them. With the help of friends, I learn to use list\_entry((\&node , struct task\_struct, sibling) to transfer them.




\section{Problem2}
\subsection{Files}
\subsubsection{ptree.c}

The program will call ptree function in kernel and print the entire process tree (in DFS order) using tabs to indent children with respect to their parents.

\subsubsection{Android.mk}

The make file for project.

\subsection{Process}

At first, I used syscall 391 in the program, but I could not make it work in kernel and I can not find the problem. With the help of TAs, I changed syscall from 391 to 356 and then it worked. 

\subsection{Result}

\begin{lstlisting}
The number of task is 59!
Print start
swapper, 0, 0, 0, 1, 0, 0
	init, 1, 1, 0, 45, 2, 0
		ueventd, 45, 1, 1, 0, 61, 0
		logd, 61, 1, 1, 0, 62, 1036
		vold, 62, 1, 1, 0, 68, 0
		healthd, 68, 1, 1, 0, 69, 0
		lmkd, 69, 1, 1, 0, 70, 0
		servicemanager, 70, 1, 1, 0, 71, 1000
		surfaceflinger, 71, 1, 1, 0, 73, 1000
		qemud, 73, 1, 1, 0, 76, 0
		sh, 76, 1, 1, 0, 77, 2000
		adbd, 77, 1, 1, 197, 78, 0
			sh, 197, 1, 77, 370, 1, 0
				ptreeARM, 370, 0, 197, 0, 1, 0
		netd, 78, 1, 1, 371, 79, 0
			sh, 371, 0, 78, 0, 1, 0
		debuggerd, 79, 1, 1, 0, 80, 0
		rild, 80, 1, 1, 0, 81, 1001
		drmserver, 81, 1, 1, 0, 82, 1019
		mediaserver, 82, 1, 1, 0, 83, 1013
		installd, 83, 1, 1, 0, 84, 0
		keystore, 84, 1, 1, 0, 85, 1017
		main, 85, 1, 1, 235, 86, 0
			system_server, 235, 1, 85, 0, 1, 1000
		gatekeeperd, 86, 1, 1, 0, 89, 1000
		perfprofd, 89, 1, 1, 0, 90, 0
		fingerprintd, 90, 1, 1, 0, 122, 1000
		bootanimation, 122, 1, 1, 0, 1, 1003
	kthreadd, 2, 1, 0, 3, 0, 0
		ksoftirqd/0, 3, 1, 2, 0, 4, 0
		kworker/0:0, 4, 1, 2, 0, 5, 0
		kworker/u:0, 5, 1, 2, 0, 6, 0
		khelper, 6, 1, 2, 0, 7, 0
		sync_supers, 7, 1, 2, 0, 8, 0
		bdi-default, 8, 1, 2, 0, 9, 0
		kblockd, 9, 1, 2, 0, 10, 0
		rpciod, 10, 1, 2, 0, 11, 0
		kworker/0:1, 11, 1, 2, 0, 12, 0
		kswapd0, 12, 1, 2, 0, 13, 0
		fsnotify_mark, 13, 1, 2, 0, 14, 0
		crypto, 14, 1, 2, 0, 25, 0
		kworker/u:1, 25, 1, 2, 0, 30, 0
		mtdblock0, 30, 1, 2, 0, 35, 0
		mtdblock1, 35, 1, 2, 0, 40, 0
		mtdblock2, 40, 1, 2, 0, 41, 0
		binder, 41, 1, 2, 0, 42, 0
		deferwq, 42, 1, 2, 0, 43, 0
		kworker/u:2, 43, 1, 2, 0, 44, 0
		mmcqd/0, 44, 1, 2, 0, 47, 0
		jbd2/mtdblock0-, 47, 1, 2, 0, 48, 0
		ext4-dio-unwrit, 48, 1, 2, 0, 51, 0
		flush-31:1, 51, 1, 2, 0, 53, 0
		jbd2/mtdblock1-, 53, 1, 2, 0, 54, 0
		ext4-dio-unwrit, 54, 1, 2, 0, 57, 0
		flush-31:2, 57, 1, 2, 0, 59, 0
		jbd2/mtdblock2-, 59, 1, 2, 0, 60, 0
		ext4-dio-unwrit, 60, 1, 2, 0, 63, 0
		kworker/0:2, 63, 1, 2, 0, 94, 0
		kauditd, 94, 1, 2, 0, 0, 0
Print end


\end{lstlisting}

\section{Problem3}
\subsection{Files}
\subsubsection{process.c}

The program enerate a new process and output “StudentIDParent” with PID, then generates its children process output “StudentIDChild” with PID.

And in child process it will execute ptree.

\subsubsection{Android.mk}

The make file for project.

\subsection{Process}

I searched on the internet for how to get the process id, then I get know that getpid() function can get the id of process. And after searching the relationship of parent process and child process, I know the pid = fork() in parent process will return the child process id.

\subsection{Result}

\section{Problem3}
\subsection{Files}
\subsubsection{process.c}

The program enerate a new process and output “StudentIDParent” with PID, then generates its children process output “StudentIDChild” with PID.

And in child process it will execute ptree.

\subsubsection{Android.mk}

The make file for project.

\subsection{Process}

I searched on the internet for how to get the process id, then I get know that getpid() function can get the id of process. And after searching the relationship of parent process and child process, I know the pid = fork() in parent process will return the child process id.

\subsection{Result}

\begin{lstlisting}
517030910116 Parent pid = 701
517030910116 Child pid = 702
The number of task is 59!
Print start
swapper, 0, 0, 0, 1, 0, 0
	init, 1, 1, 0, 45, 2, 0
		ueventd, 45, 1, 1, 0, 61, 0
		logd, 61, 1, 1, 0, 62, 1036
		vold, 62, 1, 1, 0, 68, 0
		healthd, 68, 1, 1, 0, 69, 0
		lmkd, 69, 1, 1, 0, 70, 0
		servicemanager, 70, 1, 1, 0, 71, 1000
		surfaceflinger, 71, 1, 1, 0, 73, 1000
		qemud, 73, 1, 1, 0, 76, 0
		sh, 76, 1, 1, 0, 77, 2000
		adbd, 77, 1, 1, 197, 78, 0
			sh, 197, 1, 77, 370, 1, 0
				ptreeARM, 370, 0, 197, 0, 1, 0
		netd, 78, 1, 1, 371, 79, 0
			sh, 371, 0, 78, 0, 1, 0
		debuggerd, 79, 1, 1, 0, 80, 0
		rild, 80, 1, 1, 0, 81, 1001
		drmserver, 81, 1, 1, 0, 82, 1019
		mediaserver, 82, 1, 1, 0, 83, 1013
		installd, 83, 1, 1, 0, 84, 0
		keystore, 84, 1, 1, 0, 85, 1017
		main, 85, 1, 1, 235, 86, 0
			system_server, 235, 1, 85, 0, 1, 1000
		gatekeeperd, 86, 1, 1, 0, 89, 1000
		perfprofd, 89, 1, 1, 0, 90, 0
		fingerprintd, 90, 1, 1, 0, 122, 1000
		bootanimation, 122, 1, 1, 0, 1, 1003
	kthreadd, 2, 1, 0, 3, 0, 0
		ksoftirqd/0, 3, 1, 2, 0, 4, 0
		kworker/0:0, 4, 1, 2, 0, 5, 0
		kworker/u:0, 5, 1, 2, 0, 6, 0
		khelper, 6, 1, 2, 0, 7, 0
		sync_supers, 7, 1, 2, 0, 8, 0
		bdi-default, 8, 1, 2, 0, 9, 0
		kblockd, 9, 1, 2, 0, 10, 0
		rpciod, 10, 1, 2, 0, 11, 0
		kworker/0:1, 11, 1, 2, 0, 12, 0
		kswapd0, 12, 1, 2, 0, 13, 0
		fsnotify_mark, 13, 1, 2, 0, 14, 0
		crypto, 14, 1, 2, 0, 25, 0
		kworker/u:1, 25, 1, 2, 0, 30, 0
		mtdblock0, 30, 1, 2, 0, 35, 0
		mtdblock1, 35, 1, 2, 0, 40, 0
		mtdblock2, 40, 1, 2, 0, 41, 0
		binder, 41, 1, 2, 0, 42, 0
		deferwq, 42, 1, 2, 0, 43, 0
		kworker/u:2, 43, 1, 2, 0, 44, 0
		mmcqd/0, 44, 1, 2, 0, 47, 0
		jbd2/mtdblock0-, 47, 1, 2, 0, 48, 0
		ext4-dio-unwrit, 48, 1, 2, 0, 51, 0
		flush-31:1, 51, 1, 2, 0, 53, 0
		jbd2/mtdblock1-, 53, 1, 2, 0, 54, 0
		ext4-dio-unwrit, 54, 1, 2, 0, 57, 0
		flush-31:2, 57, 1, 2, 0, 59, 0
		jbd2/mtdblock2-, 59, 1, 2, 0, 60, 0
		ext4-dio-unwrit, 60, 1, 2, 0, 63, 0
		kworker/0:2, 63, 1, 2, 0, 94, 0
		kauditd, 94, 1, 2, 0, 0, 0
Print end

\end{lstlisting}
\section{Problem4}
\subsection{Files}
\subsubsection{server.c}

Function int main()

The server will listen to the port. If there is a client want to connect, it will connect to the client and create a new thread to call serve function.

Function void *serve(void *clientfd)

It will judge whether server could serve client or not. If the count of client that server served concurrently is less than two, the server can receive the client message and change it. Then it will send the message changed to the client.

\subsubsection{client.c}

The program client.c will send message to server and receive the message changed by server.

\subsection{Process}

1. I met a lot of problems in server problem. At first, I don't know the how to create thread and how to deal with critical section problem. I find answers on the internet and use pthread and mutex.

2. I use two mutex. One mutex is for variable count which is the number of client which are served by server. The other mutex is to make new coming clients to wait if there are two clients served concurrently.

\subsection{Result}

When client1 and client2 are served by server, client3 need to wait.

\includegraphics[scale=0.3]{1.png}

When client1 quit and end the service, the client3 will be served by server.

\includegraphics[scale=0.3]{2.png}

\end{document}    % Main ?????????????????????????????????????????items 
%%%%%%%%%%%%%%%%%%%%%%%%%%%%%%%%%%%%%%%%%%%%%%%%%%%%%%%%%%%%%%%%%%%%%
%                                                                  %
%   Modified by Bing Hsu <hello@antinucleon.com> 2013              %
%   Forked From (c) 2010 - 2011 Caspar Zhang <casparant@gmail.com> %
%                                                                  %
%   This copyrighted material is made available to anyone wishing  %
%   to use, modify, copy, or redistribute it subject to the terms  %
%   and conditions of the GNU General Public License version 2.    %
%                                                                  %
%   This program is distributed in the hope that it will be        %
%   useful, but WITHOUT ANY WARRANTY; without even the implied     %
%   warranty of MERCHANTABILITY or FITNESS FOR A PARTICULAR        %
%   PURPOSE. See the GNU General Public License for more details.  %
%                                                                  %
%   You should have received a copy of the GNU General Public      %
%   License along with this program; if not, write to the Free     %
%   Software Foundation, Inc., 51 Franklin Street, Fifth Floor,    %
%   Boston, MA 02110-1301, USA.                                    %
%                                                                  %
%%%%%%%%%%%%%%%%%%%%%%%%%%%%%%%%%%%%%%%%%%%%%%%%%%%%%%%%%%%%%%%%%%%%

%%%%%%%%%%%%%%%%%%%%%%%%%%%%%%%%%%%%%%%%%%%%%%%%%%%%%%%%%%%%%%%%%%%%
%                                                                  %
%   Copyright (c) 2010 - 2011 Caspar Zhang <casparant@gmail.com>   %
%                                                                  %
%   This copyrighted material is made available to anyone wishing  %
%   to use, modify, copy, or redistribute it subject to the terms  %
%   and conditions of the GNU General Public License version 2.    %
%                                                                  %
%   This program is distributed in the hope that it will be        %
%   useful, but WITHOUT ANY WARRANTY; without even the implied     %
%   warranty of MERCHANTABILITY or FITNESS FOR A PARTICULAR        %
%   PURPOSE. See the GNU General Public License for more details.  %
%                                                                  %
%   You should have received a copy of the GNU General Public      %
%   License along with this program; if not, write to the Free     %
%   Software Foundation, Inc., 51 Franklin Street, Fifth Floor,    %
%   Boston, MA 02110-1301, USA.                                    %
%                                                                  %
%%%%%%%%%%%%%%%%%%%%%%%%%%%%%%%%%%%%%%%%%%%%%%%%%%%%%%%%%%%%%%%%%%%%

% 你只需要修改下面内容就可以完成中英文摘要,
% 这要求你具有一定的LaTeX基础,但是还是那句话,
% 如果你足够聪明,不具有LaTeX基础也可以完成。

% 中文摘要
\def\abstractcn{
%从这里开始写你的摘要,分段需要空一行。
以MOOC为主的在线教育平台席卷全球,而当前国内外对于MOOC学习的主要群体——高校大学生的学习行为研究较少。经过
观察,大学生多是在上大学之后才开始使用MOOC,而在大学之前的学习阶段不曾使用MOOC进行学习。本文自编调查问卷,
并进行了样本特征分析、信度分析、效度检验。根据问卷调查分析
统计,文章总结出影响大学生在上大学前就开始使用MOOC的影响因素
主要有家庭经济状况、高中所在学校的性质、母亲受教育程度、父亲受教育程度。
通过对这四个影响因素的权重分析,文章指出家庭经济状况和高中学校性质
对大学生在上大学前就开始使用MOOC的影响较大。最后,文章对得出的结果进行分析,提出一些增加MOOC在中小学生群体中
使用情况的建议,以期让更多的学习者受益于优质的MOOC课程资源。


%摘要结束
}
\def\abstractcns{
随着信息技术的快速发展,教育技术也在不断变革,并给教育带来新的变化。MOOC以互联网为载体,将全球顶级大学的
优质课程资源,以极低的成本传递到原本无法取得这些资源的 世界各地学习者的终端设备上,使他们随时随地获取最优
质的学习资源。此外,MOOC学习者 可以根据时间、能力自行把握学习进度,选择学习环境,充分体现了“自主学习”的理念。

据教育部在线教育研究中心发布的《2016 中国慕课行业研究白皮书》预计,2016年中国MOOC学习者将超过 1000 万,
MOOC学习者数量呈现出快速增长趋势[。然而,尽管MOOC蓬勃发展、用户数量持续增多,但相关研究显示,MOOC的使用
者主要为高校大学生,很少的部分为中小学生。为此,本研究采用问卷调查、主成分分析等研究方法,对影响大学生在上大学前
就开始使用MOOC的影响因素进行研究,并在此基础上,提出一些增加MOOC在中小学生群体中
使用情况的建议。
}

% 中文关键字 
% TODO: 改成可变长度的
\def\abscnkeyone{MOOC}
\def\abscnkeytwo{数据分析}
\def\abscnkeythree{大学生}



% 中文摘要
\begin{titlepage}
    \begin{spacing}{1.05}
        \centering
        \parbox[c]{.6\textwidth}{\thesistitlefont{\thesistitle}}
    \end{spacing}

    \begin{spacing}{1.6}
        \centering
        \sanhao\quad{} \\ 
        \abscnname{517030910116\quad{}\quad{}\quad{}席经纬\quad{}\quad{}\quad{}\quad{}F1703301} \\ 
        \xiaosanhao\quad{}
    \end{spacing}


    \begin{spacing}{0.8}
        \centering
        \sanhao\quad{} \\ 
        \abscnname{摘\quad{}要} \\ 
        \xiaosanhao\quad{}
    \end{spacing}
    
    \normalsize

    \abstractcn
    
    \quad{}

    \abscnkey{关键词}\quad{}%
    \abscnkeys{\abscnkeyone\quad{}%
                      \abscnkeytwo\quad{}%
                      \abscnkeythree\quad{}%
                      }%

    \begin{spacing}{1.6}
        \centering
        \sanhao\quad{} \\ 
        \abscnname{引\quad{}言} \\ 
        \xiaosanhao\quad{}
    \end{spacing}
    
    \normalsize

    \abstractcns
    
    \quad{}

\end{titlepage}
  % Abstract
%%%%%%%%%%%%%%%%%%%%%%%%%%%%% Main Area %%%%%%%%%%%%%%%%%%%%%%%%%%%%
\setlength{\oddsidemargin}{-.5cm}  % 3.17cm - 1 inch
\setlength{\evensidemargin}{\oddsidemargin}
\setlength{\textwidth}{17.00cm}

\section{Problem1}
\subsection{Files}
\subsubsection{ptree.c}
Function static int ptree(struct prinfo *buf, int *nr)

Call function $dfs$ and use read\_lock and read\_unlock to protect the data.Then copy data from task[] to buf.

Function void $dfs$(struct task\_struct start, int deep)

Search all processes and store information in $dfs$ order.

\subsubsection{Makefile}

Use Makefile to compile the ptree.c file.

\subsection{Process}

a. I searched a lot of information of task\_struct and list function in Linux kernel on the internet. Then I know how to use the function list\_entry and list\_for\_each.

b. When I coded the $dfs$ function, I met many problems. I don't know how to depth-first-search a tree without knowing all the children of parent. Then I search on Google and I find node -> children will point to the head of children list. And I use list\_for\_each function to depth-first-search the process tree.

c. At problem1, I don't know the difference between struct task\_struct and list\_head. And I didn't know how to transfer them. With the help of friends, I learn to use list\_entry((\&node , struct task\_struct, sibling) to transfer them.




\section{Problem2}
\subsection{Files}
\subsubsection{ptree.c}

The program will call ptree function in kernel and print the entire process tree (in DFS order) using tabs to indent children with respect to their parents.

\subsubsection{Android.mk}

The make file for project.

\subsection{Process}

At first, I used syscall 391 in the program, but I could not make it work in kernel and I can not find the problem. With the help of TAs, I changed syscall from 391 to 356 and then it worked. 

\subsection{Result}

\begin{lstlisting}
The number of task is 59!
Print start
swapper, 0, 0, 0, 1, 0, 0
	init, 1, 1, 0, 45, 2, 0
		ueventd, 45, 1, 1, 0, 61, 0
		logd, 61, 1, 1, 0, 62, 1036
		vold, 62, 1, 1, 0, 68, 0
		healthd, 68, 1, 1, 0, 69, 0
		lmkd, 69, 1, 1, 0, 70, 0
		servicemanager, 70, 1, 1, 0, 71, 1000
		surfaceflinger, 71, 1, 1, 0, 73, 1000
		qemud, 73, 1, 1, 0, 76, 0
		sh, 76, 1, 1, 0, 77, 2000
		adbd, 77, 1, 1, 197, 78, 0
			sh, 197, 1, 77, 370, 1, 0
				ptreeARM, 370, 0, 197, 0, 1, 0
		netd, 78, 1, 1, 371, 79, 0
			sh, 371, 0, 78, 0, 1, 0
		debuggerd, 79, 1, 1, 0, 80, 0
		rild, 80, 1, 1, 0, 81, 1001
		drmserver, 81, 1, 1, 0, 82, 1019
		mediaserver, 82, 1, 1, 0, 83, 1013
		installd, 83, 1, 1, 0, 84, 0
		keystore, 84, 1, 1, 0, 85, 1017
		main, 85, 1, 1, 235, 86, 0
			system_server, 235, 1, 85, 0, 1, 1000
		gatekeeperd, 86, 1, 1, 0, 89, 1000
		perfprofd, 89, 1, 1, 0, 90, 0
		fingerprintd, 90, 1, 1, 0, 122, 1000
		bootanimation, 122, 1, 1, 0, 1, 1003
	kthreadd, 2, 1, 0, 3, 0, 0
		ksoftirqd/0, 3, 1, 2, 0, 4, 0
		kworker/0:0, 4, 1, 2, 0, 5, 0
		kworker/u:0, 5, 1, 2, 0, 6, 0
		khelper, 6, 1, 2, 0, 7, 0
		sync_supers, 7, 1, 2, 0, 8, 0
		bdi-default, 8, 1, 2, 0, 9, 0
		kblockd, 9, 1, 2, 0, 10, 0
		rpciod, 10, 1, 2, 0, 11, 0
		kworker/0:1, 11, 1, 2, 0, 12, 0
		kswapd0, 12, 1, 2, 0, 13, 0
		fsnotify_mark, 13, 1, 2, 0, 14, 0
		crypto, 14, 1, 2, 0, 25, 0
		kworker/u:1, 25, 1, 2, 0, 30, 0
		mtdblock0, 30, 1, 2, 0, 35, 0
		mtdblock1, 35, 1, 2, 0, 40, 0
		mtdblock2, 40, 1, 2, 0, 41, 0
		binder, 41, 1, 2, 0, 42, 0
		deferwq, 42, 1, 2, 0, 43, 0
		kworker/u:2, 43, 1, 2, 0, 44, 0
		mmcqd/0, 44, 1, 2, 0, 47, 0
		jbd2/mtdblock0-, 47, 1, 2, 0, 48, 0
		ext4-dio-unwrit, 48, 1, 2, 0, 51, 0
		flush-31:1, 51, 1, 2, 0, 53, 0
		jbd2/mtdblock1-, 53, 1, 2, 0, 54, 0
		ext4-dio-unwrit, 54, 1, 2, 0, 57, 0
		flush-31:2, 57, 1, 2, 0, 59, 0
		jbd2/mtdblock2-, 59, 1, 2, 0, 60, 0
		ext4-dio-unwrit, 60, 1, 2, 0, 63, 0
		kworker/0:2, 63, 1, 2, 0, 94, 0
		kauditd, 94, 1, 2, 0, 0, 0
Print end


\end{lstlisting}

\section{Problem3}
\subsection{Files}
\subsubsection{process.c}

The program enerate a new process and output “StudentIDParent” with PID, then generates its children process output “StudentIDChild” with PID.

And in child process it will execute ptree.

\subsubsection{Android.mk}

The make file for project.

\subsection{Process}

I searched on the internet for how to get the process id, then I get know that getpid() function can get the id of process. And after searching the relationship of parent process and child process, I know the pid = fork() in parent process will return the child process id.

\subsection{Result}

\section{Problem3}
\subsection{Files}
\subsubsection{process.c}

The program enerate a new process and output “StudentIDParent” with PID, then generates its children process output “StudentIDChild” with PID.

And in child process it will execute ptree.

\subsubsection{Android.mk}

The make file for project.

\subsection{Process}

I searched on the internet for how to get the process id, then I get know that getpid() function can get the id of process. And after searching the relationship of parent process and child process, I know the pid = fork() in parent process will return the child process id.

\subsection{Result}

\begin{lstlisting}
517030910116 Parent pid = 701
517030910116 Child pid = 702
The number of task is 59!
Print start
swapper, 0, 0, 0, 1, 0, 0
	init, 1, 1, 0, 45, 2, 0
		ueventd, 45, 1, 1, 0, 61, 0
		logd, 61, 1, 1, 0, 62, 1036
		vold, 62, 1, 1, 0, 68, 0
		healthd, 68, 1, 1, 0, 69, 0
		lmkd, 69, 1, 1, 0, 70, 0
		servicemanager, 70, 1, 1, 0, 71, 1000
		surfaceflinger, 71, 1, 1, 0, 73, 1000
		qemud, 73, 1, 1, 0, 76, 0
		sh, 76, 1, 1, 0, 77, 2000
		adbd, 77, 1, 1, 197, 78, 0
			sh, 197, 1, 77, 370, 1, 0
				ptreeARM, 370, 0, 197, 0, 1, 0
		netd, 78, 1, 1, 371, 79, 0
			sh, 371, 0, 78, 0, 1, 0
		debuggerd, 79, 1, 1, 0, 80, 0
		rild, 80, 1, 1, 0, 81, 1001
		drmserver, 81, 1, 1, 0, 82, 1019
		mediaserver, 82, 1, 1, 0, 83, 1013
		installd, 83, 1, 1, 0, 84, 0
		keystore, 84, 1, 1, 0, 85, 1017
		main, 85, 1, 1, 235, 86, 0
			system_server, 235, 1, 85, 0, 1, 1000
		gatekeeperd, 86, 1, 1, 0, 89, 1000
		perfprofd, 89, 1, 1, 0, 90, 0
		fingerprintd, 90, 1, 1, 0, 122, 1000
		bootanimation, 122, 1, 1, 0, 1, 1003
	kthreadd, 2, 1, 0, 3, 0, 0
		ksoftirqd/0, 3, 1, 2, 0, 4, 0
		kworker/0:0, 4, 1, 2, 0, 5, 0
		kworker/u:0, 5, 1, 2, 0, 6, 0
		khelper, 6, 1, 2, 0, 7, 0
		sync_supers, 7, 1, 2, 0, 8, 0
		bdi-default, 8, 1, 2, 0, 9, 0
		kblockd, 9, 1, 2, 0, 10, 0
		rpciod, 10, 1, 2, 0, 11, 0
		kworker/0:1, 11, 1, 2, 0, 12, 0
		kswapd0, 12, 1, 2, 0, 13, 0
		fsnotify_mark, 13, 1, 2, 0, 14, 0
		crypto, 14, 1, 2, 0, 25, 0
		kworker/u:1, 25, 1, 2, 0, 30, 0
		mtdblock0, 30, 1, 2, 0, 35, 0
		mtdblock1, 35, 1, 2, 0, 40, 0
		mtdblock2, 40, 1, 2, 0, 41, 0
		binder, 41, 1, 2, 0, 42, 0
		deferwq, 42, 1, 2, 0, 43, 0
		kworker/u:2, 43, 1, 2, 0, 44, 0
		mmcqd/0, 44, 1, 2, 0, 47, 0
		jbd2/mtdblock0-, 47, 1, 2, 0, 48, 0
		ext4-dio-unwrit, 48, 1, 2, 0, 51, 0
		flush-31:1, 51, 1, 2, 0, 53, 0
		jbd2/mtdblock1-, 53, 1, 2, 0, 54, 0
		ext4-dio-unwrit, 54, 1, 2, 0, 57, 0
		flush-31:2, 57, 1, 2, 0, 59, 0
		jbd2/mtdblock2-, 59, 1, 2, 0, 60, 0
		ext4-dio-unwrit, 60, 1, 2, 0, 63, 0
		kworker/0:2, 63, 1, 2, 0, 94, 0
		kauditd, 94, 1, 2, 0, 0, 0
Print end

\end{lstlisting}
\section{Problem4}
\subsection{Files}
\subsubsection{server.c}

Function int main()

The server will listen to the port. If there is a client want to connect, it will connect to the client and create a new thread to call serve function.

Function void *serve(void *clientfd)

It will judge whether server could serve client or not. If the count of client that server served concurrently is less than two, the server can receive the client message and change it. Then it will send the message changed to the client.

\subsubsection{client.c}

The program client.c will send message to server and receive the message changed by server.

\subsection{Process}

1. I met a lot of problems in server problem. At first, I don't know the how to create thread and how to deal with critical section problem. I find answers on the internet and use pthread and mutex.

2. I use two mutex. One mutex is for variable count which is the number of client which are served by server. The other mutex is to make new coming clients to wait if there are two clients served concurrently.

\subsection{Result}

When client1 and client2 are served by server, client3 need to wait.

\includegraphics[scale=0.3]{1.png}

When client1 quit and end the service, the client3 will be served by server.

\includegraphics[scale=0.3]{2.png}

\end{document}    % Main ?????????????????????????????????????????items 
%%%%%%%%%%%%%%%%%%%%%%%%%%%%%%%%%%%%%%%%%%%%%%%%%%%%%%%%%%%%%%%%%%%%%
%                                                                  %
%   Modified by Bing Hsu <hello@antinucleon.com> 2013              %
%   Forked From (c) 2010 - 2011 Caspar Zhang <casparant@gmail.com> %
%                                                                  %
%   This copyrighted material is made available to anyone wishing  %
%   to use, modify, copy, or redistribute it subject to the terms  %
%   and conditions of the GNU General Public License version 2.    %
%                                                                  %
%   This program is distributed in the hope that it will be        %
%   useful, but WITHOUT ANY WARRANTY; without even the implied     %
%   warranty of MERCHANTABILITY or FITNESS FOR A PARTICULAR        %
%   PURPOSE. See the GNU General Public License for more details.  %
%                                                                  %
%   You should have received a copy of the GNU General Public      %
%   License along with this program; if not, write to the Free     %
%   Software Foundation, Inc., 51 Franklin Street, Fifth Floor,    %
%   Boston, MA 02110-1301, USA.                                    %
%                                                                  %
%%%%%%%%%%%%%%%%%%%%%%%%%%%%%%%%%%%%%%%%%%%%%%%%%%%%%%%%%%%%%%%%%%%%

%%%%%%%%%%%%%%%%%%%%%%%%%%%%%%%%%%%%%%%%%%%%%%%%%%%%%%%%%%%%%%%%%%%%
%                                                                  %
%   Copyright (c) 2010 - 2011 Caspar Zhang <casparant@gmail.com>   %
%                                                                  %
%   This copyrighted material is made available to anyone wishing  %
%   to use, modify, copy, or redistribute it subject to the terms  %
%   and conditions of the GNU General Public License version 2.    %
%                                                                  %
%   This program is distributed in the hope that it will be        %
%   useful, but WITHOUT ANY WARRANTY; without even the implied     %
%   warranty of MERCHANTABILITY or FITNESS FOR A PARTICULAR        %
%   PURPOSE. See the GNU General Public License for more details.  %
%                                                                  %
%   You should have received a copy of the GNU General Public      %
%   License along with this program; if not, write to the Free     %
%   Software Foundation, Inc., 51 Franklin Street, Fifth Floor,    %
%   Boston, MA 02110-1301, USA.                                    %
%                                                                  %
%%%%%%%%%%%%%%%%%%%%%%%%%%%%%%%%%%%%%%%%%%%%%%%%%%%%%%%%%%%%%%%%%%%%

% 你只需要修改下面内容就可以完成中英文摘要,
% 这要求你具有一定的LaTeX基础,但是还是那句话,
% 如果你足够聪明,不具有LaTeX基础也可以完成。

% 中文摘要
\def\abstractcn{
%从这里开始写你的摘要,分段需要空一行。
以MOOC为主的在线教育平台席卷全球,而当前国内外对于MOOC学习的主要群体——高校大学生的学习行为研究较少。经过
观察,大学生多是在上大学之后才开始使用MOOC,而在大学之前的学习阶段不曾使用MOOC进行学习。本文自编调查问卷,
并进行了样本特征分析、信度分析、效度检验。根据问卷调查分析
统计,文章总结出影响大学生在上大学前就开始使用MOOC的影响因素
主要有家庭经济状况、高中所在学校的性质、母亲受教育程度、父亲受教育程度。
通过对这四个影响因素的权重分析,文章指出家庭经济状况和高中学校性质
对大学生在上大学前就开始使用MOOC的影响较大。最后,文章对得出的结果进行分析,提出一些增加MOOC在中小学生群体中
使用情况的建议,以期让更多的学习者受益于优质的MOOC课程资源。


%摘要结束
}
\def\abstractcns{
随着信息技术的快速发展,教育技术也在不断变革,并给教育带来新的变化。MOOC以互联网为载体,将全球顶级大学的
优质课程资源,以极低的成本传递到原本无法取得这些资源的 世界各地学习者的终端设备上,使他们随时随地获取最优
质的学习资源。此外,MOOC学习者 可以根据时间、能力自行把握学习进度,选择学习环境,充分体现了“自主学习”的理念。

据教育部在线教育研究中心发布的《2016 中国慕课行业研究白皮书》预计,2016年中国MOOC学习者将超过 1000 万,
MOOC学习者数量呈现出快速增长趋势[。然而,尽管MOOC蓬勃发展、用户数量持续增多,但相关研究显示,MOOC的使用
者主要为高校大学生,很少的部分为中小学生。为此,本研究采用问卷调查、主成分分析等研究方法,对影响大学生在上大学前
就开始使用MOOC的影响因素进行研究,并在此基础上,提出一些增加MOOC在中小学生群体中
使用情况的建议。
}

% 中文关键字 
% TODO: 改成可变长度的
\def\abscnkeyone{MOOC}
\def\abscnkeytwo{数据分析}
\def\abscnkeythree{大学生}



% 中文摘要
\begin{titlepage}
    \begin{spacing}{1.05}
        \centering
        \parbox[c]{.6\textwidth}{\thesistitlefont{\thesistitle}}
    \end{spacing}

    \begin{spacing}{1.6}
        \centering
        \sanhao\quad{} \\ 
        \abscnname{517030910116\quad{}\quad{}\quad{}席经纬\quad{}\quad{}\quad{}\quad{}F1703301} \\ 
        \xiaosanhao\quad{}
    \end{spacing}


    \begin{spacing}{0.8}
        \centering
        \sanhao\quad{} \\ 
        \abscnname{摘\quad{}要} \\ 
        \xiaosanhao\quad{}
    \end{spacing}
    
    \normalsize

    \abstractcn
    
    \quad{}

    \abscnkey{关键词}\quad{}%
    \abscnkeys{\abscnkeyone\quad{}%
                      \abscnkeytwo\quad{}%
                      \abscnkeythree\quad{}%
                      }%

    \begin{spacing}{1.6}
        \centering
        \sanhao\quad{} \\ 
        \abscnname{引\quad{}言} \\ 
        \xiaosanhao\quad{}
    \end{spacing}
    
    \normalsize

    \abstractcns
    
    \quad{}

\end{titlepage}
  % Abstract
%%%%%%%%%%%%%%%%%%%%%%%%%%%%% Main Area %%%%%%%%%%%%%%%%%%%%%%%%%%%%
\setlength{\oddsidemargin}{-.5cm}  % 3.17cm - 1 inch
\setlength{\evensidemargin}{\oddsidemargin}
\setlength{\textwidth}{17.00cm}
\CJKfamily{song}

\section{Problem1}
\subsection{Files}
\subsubsection{ptree.c}
Function static int ptree(struct prinfo *buf, int *nr)

Call function $dfs$ and use read\_lock and read\_unlock to protect the data. Then copy data from ask to buf.

Function void $dfs$(struct task\_struct start, int deep)

Search all processes and store information in $dfs$ order.

\subsubsection{Makefile}

Use Makefile to compile the ptree.c file.

\subsection{Process}


\section{共享经济案例分析——以 ofo 为例}
\subsection{共享单车ofo发展历程}
\subsubsection{共享单车ofo的兴起}
起初,ofo 只是个小而美的校园创业项目,2015 年 6 月,北大研究生戴威自掏腰包采购 200 辆小黄车投放在北大校园,并在校园推出“共享计划”,向学生回收单车作为共享单车。这样用户共享了一辆车,就能获得所有小黄车的使用权。

当时戴威宣称要在北大内推出 10000 辆共享单车,并面向北大师生招募 2000 位共享车主,这时候的共享单车还算是名副其实的共享经济。

ofo 这一共享单车模式也很快得到认可,10 月份在北大校园日均订单已经有 4000 单,并获得了第一笔 900 万的融资。

\subsubsection{共享单车ofo的发展}
 2016 年 ofo 逐渐向全国 20 多个城市的 200 多所高校推广,在校园里积累了 80 万用户,日均订单达到 20 万。

直到 2016 年10 月,ofo 正式走出校园 ,进军城市市场。背后是大量资本的急速推动,仅在 2016 年几个月时间里,ofo 就经历了 5 轮融资,累计融资金额超过了 2 亿美元。

2015,2016,2017三年间,ofo累计获得了12轮融资(包括2轮战略融资),平均3个月完成一轮。有数据显示,截至2018年1月,ofo估值为30亿美元。这个数字较2016年,翻了200倍。

与疯狂的融资速度形成呼应,ofo的业务也在疯狂扩张,成为继淘宝、京东、滴滴等巨头之后,中国第9家日订单过百万的互联网平台。

同年12月,ofo开始在美国旧金山、英国伦敦、新加坡运营,而后又进入马来西亚、日本、西班牙、以色列等国。
ofo官网数据显示,这家共享单车平台已发展成为全球规模最大的共享出行平台,全球连接超1000万辆共享单车,日订单超3200万。

\subsubsection{共享单车ofo的衰落}

从2017年下半年开始,随着共享单车资本泡沫的逐渐破裂,一个个曾经辉煌一时的公司或倒下、或卖身,这让本来就如履薄冰的ofo,雪上加霜。

2017年12月,有新闻称,ofo挪用30亿元用户押金,账面上可供调配的资金仅剩3.5亿元。2018年1月,又有新闻称,ofo账户可用资金剩不足6亿元以及ofo还可能拖欠供应商约25亿元的货款。

之后,ofo采取了一系列“节流”措施,可以看出,这家公司已经非常需要资金了。

一方面,ofo取消了全国20个城市的芝麻信用免押金活动,改为“95元充余额免押”;另一方面,因难从用户的单次骑行获利,ofo开始发动员工卖车身广告,转向广告端寻求盈利。

2018年6月份,虎嗅在《小黄车快黄了》一文中指出,ofo总部正在“上演”管理层剧变和史上最大规模裁员。

2018年10月22日,ofo的运营公司法定代表人突然发生变更。陈正江替代戴威,成为东峡大通(北京)管理咨询有限公司的法定代表人。

总之,现在小黄车真的要“黄”了。

\subsection{共享单车ofo兴起原因}

\subsubsection{满足客户需求}

共享单车ofo从建立之初便坚持市场营销观念,以满足客户需求为出发点,强调“顾客需要什么,就生产什么”,准确确定目标市场的需要和欲望,很好的抓住了客户出行“最后一公里”的需求。之前,对于公众提升“最后一公里”出行效率的需求,有关方面一直在探索,也推出了一些解决方式,但接受度并不是很高。以北京为例,在很多地铁站和居民小区之间,运行着社区微循环电瓶车。但这些电瓶车要么发车时间固定,要么需要等坐满人才发车。同时,相比运送距离,每次两三元钱的乘车价格也让不少人觉得难以接受,这些不足都降低了乘客的搭乘意愿和乘坐体验。与这些出行方式相比,共享单车ofo一出现,便迅速成为很多人的首选。其开锁、上锁以及支付等所有程序,可以用一部手机全部完成,这让用户在使用共享单车ofo时,“随骑随走”和“随停随锁”成为可能,最大程度地覆盖了用户较为随机的“最后一公里”路线。

\subsubsection{把握目标市场}
共享单车ofo在市场定位时准确把握自己的目标市场是普通大众,精确找到自己在细分市场中的位置。当时,共享经济才刚刚开始发展,在交通领域共享单车竞争压力较小,所以共享单车ofo一出现便迅速占领了市场,发展十分迅速。

另外,共享单车ofo准确认识到自己的优势在于满足人们“最后一公里”的需求。例如,校园是典型的“最后一公里”场景,这里没有公交,不鼓励汽车,自行车就是师生们最好的代步工具。ofo就是以大学校园为发起地,后来ofo在校园市场打下根基后,逐步将触角延伸至城市共享。中国有数以千万计的高校在校学生,每年数百万的新生,校园市场的成长空间非常显著;另一个场景是3到5公里的短途出行,这是打车软件和传统公共交通的“盲区”,更是共享单车们争夺市场的焦点。

\subsubsection{价格与促销策略}

共享单车ofo在推出时便根据自己成本和用户的需求将定价确定为1元/每小时,这样的定价让许多人都可以接受,扩大了用户群体。同时,共享单车ofo还十分注重推销策略,利用广告这一带有浓郁商业性的艺术方式向大众推广自己的产品。如共享单车ofo联手环球影业推出“小黄人ofo”,使用“小黄人”系列电影的IP形象,借着《神偷奶爸3》即将上映的东风,这样在营销中增添了“玩”的元素,一方面能让活动变得更有趣,但更重要的是能增强受众的浸入感,提高了用户体验。


\subsection{共享单车ofo衰落原因}

\subsubsection{过度投放引起社会问题}
随着共享单车行
业的竞争越来越激烈,各大平台争相加大投放量争
夺市场份额。由于用户不文明使用共享单车的情况
屡禁不止,单车损毁率高,平台的维护能力跟不上
单车损毁速度,也大大增加了平台大量投放新单车
的速度。多数共享单车都面临着在公共场所如何停
放的问题,尽管单辆共享单车所占空间不大,但庞
大的数量和无序的摆放, 不利于城市交通管理,给
城市造成了一定的负担。

\subsubsection{平台信息安全性差}
注册ofo 时,用户需要
实名认证,提供的个人信息有手机号、身份证号等,
而且,ofo 后台也会产生用户平时的骑行路径、家或
工作单位的具体地址等。共享单车ofo平台没有对用户
的信息进行很好的保护,导致ofo的注册用户信息很容易泄露。


\subsubsection{平台盈利能力弱}

上海凤凰是ofo 单车的上游供应商, 根据凤凰2017 年年报,ofo 单车在2017年共采购车辆178 万辆, 确定收入为59627 万,所以可知一辆ofo 单车制造成本为335 元。按单车日使用次数加上投入资本相对较低,缺乏车体加固且采用链条驱动方式,增加了链条脱落的风险,毁损率较高,大约在18\%~20\%,所以其运营成本与维护成本就会相对较高。据杭州市城管委数据,每辆单车的修理费用在3.7 至6.7 元间, 搬走一辆车需要的人工费用是9.6 元,再加上早期的ofo 采用的是传统锁,缺乏导航定位系统,也无形中增加了人工寻找已报修的ofo 的难度。这些因素导致共享单车的盈利能力赶不上制造成本和维修费用的消耗,其他融资路径一旦断绝,共享单车企业便陷入困境。


\subsubsection{企业内部管理体系缺失}
由于共享单车ofo公司内部管理体系的缺失,ofo创始人戴威在ofo的股权被架空,在ofo股权架构中,戴威占股比为36.02
\%,滴滴占股比为25.32\%,而经纬中国、金沙江、王刚等,实际上都是滴滴的投资方,可归类为滴滴系,他们的股权加起来或将远超戴威。
同时,ofo的内部区域经理有贪腐现象,区域经理每月通过虚报修车数量而套取公司大量资金,数额从数万到几十万元不等。而ofo公司方面却没有对此进行任何调查。

\section{共享经济发展建议}

通过分析共享单车ofo商业
模式的兴起与衰落原因,得出共享产业发展中存在的
过度共享等一系列社会问
题,用户信息泄露等法律问题,以及
盈利能力弱、企业内部管理体系缺失等问题。下面给出一些促进共享经济健康发展的建议:

\subsection{企业完善平台}
共享单车企业应改进平台技术缺陷,加强对共享单车的治理。
平台首先应加强自身的技术建设,ofo 应做好技术改造,
如:将机械锁更换为智能锁,设置网络技术安全指
标以保护注册用户的隐私等;其次应控制共享单车
的投放量,据报道,共享单车的投放规模已接近饱
和点,所以接下来应注重提高分享率而不是一味地
扩大投放量;同时,应加强与政府合作,积极投身于
共享单车的治理。
\subsection{政府加强管理}
共享产品被私人化占有与恶意损毁的现象屡禁不止, 从侧面反映
出我国市场经济还没有充分成熟、不能及时地解决
社会问题及我国现阶段诚信体系的建立还相对滞
后。所以,我国政府一是应该
完善相关法律体系, 解决由于用户丧失诚信引发的
恶意损坏、私人占有的现象,加大对共享经济创新支
持力度,形成有利于其发展的良好环境。二是应建立
诚信体系,完善国民信用体系建设,加强各信用平台
的数据共享, 将用户不文明使用共享单车的行为纳
入社会的诚信体系,可以采取扣分制或黑名单制,提
高违约成本, 也可激励用户规范自身行为的同时也
参与到监管中,举报其他人的违规行为。
\subsection{健全管理体系}
共享经济企业应建立健全自己公司内部的管理体系,
杜绝贪腐现象的发生,提高员工工作效率和积极性。
另外,企业应该建立押金托管制度,合理管理资金池。共享产品一般一物对多人,
用户在缴存押金后,押金直接转移
到共享平台,用户无法得知押金的使用流向及
是否有第三方监管,这就给平台侵占、挪用用户资
金创造了条件。在这方面,政府部门应该设立监管部门,
利用独立第三方机构监管押金的流向,从而避免发
生资金沉淀和被挪用的风险。






%%%%%%%%%%%%%%%%%%%%%%% Main Area ENDs Here %%%%%%%%%%%%%%%%%%%%%%%%
\addcontentsline{toc}{chapter}{参考文献}
\begin{thebibliography}{00}
  \bibitem{r1}王玲 . 共享单车商业模式探讨及建议——以ofo 共享单车为例.《时代金融》, 2018.10
  \bibitem{r2} 周子晗.共享经济商业模式的风险与盈利方式研究. 《江苏商论》,2018.12
  \bibitem{r3} 王天一,康朔腾. 共享单车的发展方向和盈利方式研究. 《市场研究》, 2018.10
   \bibitem{r4}姚珣,乔俐萌 .互联网背景下的“共享经济模式”之争——基于Uber和共享单车研究.《电子科技大学学报》, 2018.12
   \bibitem{r5}高琦. 我国共享经济的发展和培育现状及问题研究 . 《辽宁行政学院学报 》,2018.6
\end{thebibliography}

\end{document}